% Define document class
\documentclass[twocolumn]{aastex631}
\usepackage{showyourwork}

% Begin!
\begin{document}

% Title
\title{An open source scientific article}

% Author list
\author{@tobibu}

% Abstract with filler text
\begin{abstract}
    Lorem ipsum dolor sit amet, consectetuer adipiscing elit.
    Ut purus elit, vestibulum ut, placerat ac, adipiscing vitae, felis.
    Curabitur dictum gravida mauris, consectetuer id, vulputate a, magna.
    Donec vehicula augue eu neque, morbi tristique senectus et netus et.
    Mauris ut leo, cras viverra metus rhoncus sem, nulla et lectus vestibulum.
    Phasellus eu tellus sit amet tortor gravida placerat.
    Integer sapien est, iaculis in, pretium quis, viverra ac, nunc.
    Praesent eget sem vel leo ultrices bibendum.
    Aenean faucibus, morbi dolor nulla, malesuada eu, pulvinar at, mollis ac.
    Curabitur auctor semper nulla donec varius orci eget risus.
    Duis nibh mi, congue eu, accumsan eleifend, sagittis quis, diam.
    Duis eget orci sit amet orci dignissim rutrum.
\end{abstract}

% Main body with filler text
\section{Introduction}
\label{sec:intro}

Lorem ipsum dolor sit amet, consectetuer adipiscing elit.
Ut purus elit, vestibulum ut, placerat ac, adipiscing vitae, felis.
Curabitur dictum gravida mauris, consectetuer id, vulputate a, magna.
Donec vehicula augue eu neque, morbi tristique senectus et netus et.
Mauris ut leo, cras viverra metus rhoncus sem, nulla et lectus vestibulum.
Phasellus eu tellus sit amet tortor gravida placerat.
Integer sapien est, iaculis in, pretium quis, viverra ac, nunc.
Praesent eget sem vel leo ultrices bibendum.
Aenean faucibus, morbi dolor nulla, malesuada eu, pulvinar at, mollis ac.
Curabitur auctor semper nulla donec varius orci eget risus.
Duis nibh mi, congue eu, accumsan eleifend, sagittis quis, diam.
Duis eget orci sit amet orci dignissim rutrum.

Nam dui ligula, fringilla a, euismod sodales, sollici- tudin vel, wisi.
Morbi auctor lorem non justo, nam lacus libero, pretium at, lobortis vitae.
Donec aliquet, tortor sed accumsan bibendum, erat ligula aliquet magna.
Morbi ac orci et nisl hendrerit mollis, suspendisse ut massa, cras nec ante.
Pellentesque a nulla cum sociis natoque penatibus et magnis dis parturient.
Aliquam tincidunt urna, nulla ullamcorper vestibulum turpis.
Pellentesque cursus luctus mauris \citep{Luger2021}.



\begin{figure}
\begin{center}
    \includegraphics[width = 0.5\textwidth]{figures/efficiency_comparison.pdf}
    \caption{Comparison of different star formation efficiency, $\epsilon_{\rm ff}$ as a function of virial parameter $\alpha_{\rm vir}$ for the models of \citet[][blue]{Padoan_2012}, \citet[][pink]{Hopkins_2013}, \citet[][grey]{Evans_2022}, \citet[][lightgreen]{Semenov_2016} and \citet{Federrath2012} with the Mach number = 0.01 (yellow), 0.1 (darkgreen), 1.0 (red), 10.0 (purple), 100.0 (brown). \T{Shall we remove Semenov and Evans and only keep the models we actually use? Or leave Semenov and Evans as a variant of Padoan's model?} \textcolor{orange}{Maybe leave it as a variant and say that Evans leads to too low SFEs and that there is no significant difference between Padoan and Semenov, so we use Padoan and adopt Semenovs factor?}}
    \label{fig:efficiency_comparison}
    \script{efficiency_comparison.py}
\end{center}
\end{figure}

\bibliography{bib}

\end{document}
